\chapter[Aula 5]{Funções Mensuráveis e Variáveis Aleatórias}
\chaptermark{}

\section{Funções Mensuráveis}

Denotamos por $\overline{\R}$ a reta estendida, isto é, 
$\overline{\R} = \R\cup\{-\infty,+\infty\}$. Podemos 
estender naturalmente conceito de soma e produto 
para os seguintes pares de elementos 
de $\overline{\R}$ da seguinte maneira:
\begin{enumerate}
	\item 
	Se $x,y\in \R$ a soma é a soma usual 
	e o mesmo para o produto. 
	
	\item para todo $x\in \overline{R}$ temos 
	$0\cdot x=0$.
	
	\item Para todo $x\in \overline{R}$ diferente de 
	$-\infty$ definimos $x+(+\infty)=+\infty$.

	\item Para todo $x\in \overline{R}$ diferente de 
	$+\infty$ definimos $x+(-\infty)=-\infty$.
\end{enumerate}

Não vamos nos preocupar neste momento em munir $\overline{\R}$ 
de uma topologia, mas vamos definir a $\sigma$-álgebra
de Borel de $\overline{R}$,
como sendo a coleção formada pela reunião da
coleção $\mathscr{B}(\R)$ 
e de todos conjuntos da forma 
$B\cup\{-\infty\}$, $B\cup\{+\infty\}$ e $B\cup\{-\infty,+\infty\}$,
onde $B$ varia sobre todos os elementos de $\mathscr{B}(\R)$.
Esta coleção que acabamos de definir
é de fato uma $\sigma$-álgebra e será chamada de $\sigma$-álgebra
de borel de $\overline{R}$ e denotada por $\mathscr{B}(\overline{\R})$.

\begin{exercicio}
	Mostre que a coleção $\mathscr{B}(\overline{\R})$ 
	é uma $\sigma$-álgebra.
\end{exercicio}

\begin{exercicio}
	Podemos ver $\overline{R}$ como um conjunto totalmente ordenado se 
	consideramos a relação de ordem ``$<$'' obtida pela extensão natural 
	da relação de ordem em $\R$.
	Seja $\tau$ a topologia da ordem definida por ``$<$'' 
	em $\overline{R}$. Mostre que a $\sigma$-álgebra gerada pelos 
	abertos de $\tau$ coincide com a coleção $\mathscr{B}(\overline{\R})$
	definida acima. 
	
\end{exercicio}










\subsection*{Funções a Valores Reais Mensuráveis}

Nesta subseção vamos considerar funções que
saem de um espaço mensurável $(\Omega,\F)$ e toma 
valores em $\mathbb{R}$. O caso mais geral, onde
as funções assumem valores em $\overline{R}$ será
tratado na subseção seguinte.


\begin{definicao} 
Seja $(\Omega,\F)$ um espaço mensurável. 
Uma função $f:\Omega\to\R$ é dita $\F$-mensurável
se para todo número real $\alpha$ temos que
\[
	\{x\in\Omega : f(x)>\alpha \} \in \F.
\]
\end{definicao}

O próximo lema fornece três maneira alternativas de
definir funções mensuráveis. 

\begin{lema}
As seguintes afirmações são equivalentes para uma função 
$f:\Omega\to\R$.
\begin{enumerate}
	\item 
	Para todo $\alpha\in\R$ o conjunto 
	$A_{\alpha}\equiv\{x\in\Omega : f(x)>\alpha \} \in \F$.

	\item 
	Para todo $\alpha\in\R$ o conjunto 
	$B_{\alpha}\equiv\{x\in\Omega : f(x)\leq \alpha \} \in \F$.

	\item 
	Para todo $\alpha\in\R$ o conjunto 
	$C_{\alpha}\equiv\{x\in\Omega : f(x)\geq\alpha \} \in \F$.

	\item 
	Para todo $\alpha\in\R$ o conjunto 
	$D_{\alpha}\equiv\{x\in\Omega : f(x)<\alpha \} \in \F$.
	
\end{enumerate}
\end{lema}

