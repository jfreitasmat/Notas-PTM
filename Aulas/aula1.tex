\chapter[Aula 1]{Conjuntos e Limite de Sequências de Conjuntos}
\chaptermark{}

ESTE É UM TESTE.
VOU CONTINUAR O TESTE.

\section{Teoria Básica de Conjuntos}
Notações

\begin{enumerate}
\item[$\blacklozenge$] $\Omega$ espaço Amostral.
\item[$\blacklozenge$] $\mathcal{P}(\Omega)$ conjunto das partes de $\Omega$.
\item[$\blacklozenge$] $A,B$ em geral denotam subconjuntos de $\Omega$.
\item[$\blacklozenge$] $\mathcal{A}, \mathcal{B}$ coleção de subconjuntos de $\Omega$.
\item[$\blacklozenge$] $w \in \Omega$ elemento de $\Omega$.
\end{enumerate}


Operações de Conjuntos


\begin{enumerate}
\item[$\blacklozenge$] $A^{c}= \{w \in \Omega; w \notin A\}$
\item[$\blacklozenge$] Se $T$ é um conjunto arbitrário
	\begin{enumerate}
	\item[$\lozenge$]  $\displaystyle\bigcap_{t \in T}{A_t}
							= 
						\{ w \in \Omega;\ w \in A_t,\ \forall t \in T\}$.
	\item[$\lozenge$]  $\displaystyle\bigcup_{t \in T}{A_t}
							= 
						\{ w \in \Omega;\ w \in A_t,\ \text{para algum}\ t \in T \}$.
	\end{enumerate}
\end{enumerate}

Quando $A \cap B = \emptyset$, dizemos que $A$ e $B$ são disjuntos.

$A_1, A_2,\ldots$ é uma sequência mutuamente disjunta se $A_i \cap A_j = \emptyset$ 
quando $i \neq j$. Nesse caso, denotamos

$$
	\displaystyle\bigcap_{n \geqslant 1}{A_n} = \sum_{n \geqslant 1} A_n.	
$$

Outras Notações:

\begin{enumerate}
\item[$\blacklozenge$] $ A \cap B = AB$.
\item[$\blacklozenge$] $A \setminus B = A \cap B^c$.
\item[$\blacklozenge$] Diferença Simétrica $A\triangle B = (A \setminus B) \cup (B\setminus A).$
\item[$\blacklozenge$] $(A^c)^c = A $.
\item[$\blacklozenge$] $\emptyset^c = \Omega$.
\item[$\blacklozenge$] $\Omega^c = \emptyset$.
\end{enumerate}

Propriedade Associativa

$(A\cup B)\cup C= A \cup ( B \cup C)$.

$(A\cap B)\cap C= A \cap ( B \cap C)$.

\vspace*{1cm}
Leis de De Morgan

$\left( \displaystyle\bigcup_{t \in T}{A_t} \right)^c = \displaystyle\bigcap_{t \in T}{A_t^c}$.

$\left( \displaystyle\bigcap_{t \in T}{A_t} \right)^c = \displaystyle\bigcup_{t \in T}{A_t^c}$.


\vspace*{1cm}
Propriedade Distributiva

$B \cap \left( \displaystyle\bigcup_{t \in T}{A_t} \right) = \displaystyle\bigcup_{t \in T}{(B\cap A_t)} $.

$B \cup \left( \displaystyle\bigcap_{t \in T}{A_t} \right) = \displaystyle\bigcap_{t \in T}{(B\cup A_t)} $.

\vspace*{1cm}
\textbf{Função Indicadora}

Se $A \subseteq \Omega$. A função indicadora de $A$ é denotada por $1_A: \Omega \to \R$ e definida 
como segue

$$
	1_A(w) =
	\begin{cases}
		1, & \text{se}\ w \in A, \\
		0, & \text{c.c.}
	\end{cases}
$$

\begin{observacao} Usando a função indicadora, temos as relações

$1_A \leqslant 1_B \Leftrightarrow A \subseteq B$.

$1_{A^c}= 1- 1_A$.
\end{observacao}


\vspace*{1cm}
\textbf{Limite de Conjuntos}

Seja uma sequência $A_n \in \Omega$.
\begin{enumerate}
\item[$\blacklozenge$] $\inf \limits_{k\geqslant n} A_k = \displaystyle\bigcap_{k=n}^{\infty}{A_k}$.
\item[$\blacklozenge$] $\sup \limits_{k\geqslant n} A_k = \displaystyle\bigcup_{k=n}^{\infty}{A_k}$.
\item[$\blacklozenge$] 
	$\liminf \limits_{n \to \infty} A_n 
	= 
	\displaystyle\bigcup_ {n\geqslant 1} \left(\displaystyle\bigcap_{k=n}^{\infty}{A_k} \right)$.
\item[$\blacklozenge$] 
	$\limsup \limits_{n \to \infty} A_n 
	= 
	\displaystyle\bigcap_ {n\geqslant 1} \left(\displaystyle\bigcup_{k=n}^{\infty}{A_k} \right)$.
\end{enumerate}

\begin{definicao}
	Se para alguma sequência de conjuntos $B_n \subseteq \Omega$ temos

	$$
		\liminf \limits_{n \to \infty} B_n = \limsup \limits_{n \to \infty} B_n = B,
	$$	
	dizemos que existe $\lim \limits_{n \to \infty} B_n =B$.
\end{definicao}

Note que $\liminf \limits_{n\to \infty}A_n 
			=
			\lim \limits_{n\to \infty}\left(\inf \limits_{k\geqslant n}A_k \right)
		 $.


\begin{lema}
Seja $\{A_n\}$ uma sequência de subconjuntos de $\Omega$.
\begin{enumerate}
\item[a)] $\limsup A_n = \{ w \in \Omega;\ \sum_{n \geqslant 1} 1_{A_n}(w)= \infty \}$.
\item[b)] 
\begin{align*}
\vspace*{-2cm}
\liminf A_n & = \{ w \in \Omega;\ w \in A_n \text{ para todo $n$ excepto uma quantidade finita} \} \\
			& = \{w \in \Omega;\ \sum_{n\geqslant 1} 1_{A_n^c}(w) < \infty \} \\
			& = \{w \in \Omega;\ w \in A_n, \forall n \geqslant n_0(w) \}. 
\end{align*} 
\end{enumerate}
\end{lema}

\begin{proof}
Vejamos o item $(a)$. Suponha que $w \in \limsup A_n$, então $w \in \displaystyle\bigcup_{k \geq n} A_k,
\forall n \in \N$, logo existe $k_n \geqslant n$ tal que $w \in A_{k_n}$. Assim

	$$
		\sum \limits_{k \geqslant 1} 1_{A_k}(w) \geqslant \sum \limits_{n\geqslant 1} 1_{A_{k_n}}(w) 
		= 
		\infty.
	$$

Reciprocamente, se $ w \in \{ w \in \Omega;\ \sum_{n \geqslant 1} 1_{A_n}(w)= \infty \}$ daí existem 
infinitos $k$ tais que $w \in A_k$. Portanto $w \in \limsup A_n$.
\end{proof}

\begin{observacao}

\begin{enumerate}
\item[i)] $\liminf A_n \subseteq \limsup A_n$.
\item[ii)] $\left( \liminf A_n \right)^c = \limsup A_n^c$. 
\item[iii)] $\{X_n,\ n\geqslant 0\}$ sequência de v.a. No caso que desejamos mostrar que $X_n \to X$ q.c.
podemos usar o seguinte critério:

$$
	\mathbb{P}( |X_n-X| > \epsilon \ \text{infinitas vezes} )=0
$$ 
isto é, fazendo $A_n= \{ |X_n-X|> \epsilon\}$ então $\mathbb{P} (\limsup A_n )=0$.
\end{enumerate}
\end{observacao}

\vspace*{1cm}
\textbf{Sequências Monótonas}

Se $\{A_n\}$ é uma sequência de conjuntos de $\Omega$. Dizemos que $\{A_n\}$ é monótona não
decrescente se $A_1 \subseteq A_2 \subseteq \ldots$ Analogamente definimos sequência não decrescente.

Notação: $A_n \nearrow$ ou $A_n \uparrow$ (analogamente $A_n\searrow$ ou $A_n \downarrow$).

\begin{proposicao}
 Suponha que $\{A_n\}$ é uma sequência monótona.
 \begin{enumerate}
 \item[1)] Se $A_n \nearrow$ então 
 	$\exists \lim \limits_{n \to \infty} A_n= \displaystyle\bigcup_{n\geqslant 1} {A_n}$.
 \item[2)] Se $A_n \searrow$ então 
 	$\exists \lim \limits_{n \to \infty} A_n= \displaystyle\bigcap_{n\geqslant 1} {A_n}$.
 \end{enumerate}
\end{proposicao}


\begin{proof}
Vejamos a primeira parte. Queremos mostrar que 
$\liminf A_n = \limsup A_n =\displaystyle\bigcup_{n\geqslant 1} {A_n}$. Já que $A_j \subseteq A_{j+1}$
então $\displaystyle\bigcap_{k\geqslant n} {A_k}=A_n$.

Daí $\liminf A_n = \displaystyle\bigcup_{n\geqslant 1} {A_n}$. Agora 

$$
	\limsup A_n 
	=
	\displaystyle\bigcap_{n\geqslant 1} \left(\displaystyle\bigcup_{k\geqslant n}{A_k} \right)
	\subseteq 
	\displaystyle\bigcup_{k\geqslant 1} {A_k} 
	=
	\liminf A_n \subseteq \limsup A_n.
$$
Para a segunda parte é somente usar as leis de De Morgan. Isto conclui a prova.
\end{proof}

\vspace*{1cm}
\textbf{Relações de ``Dualidade"}
\begin{enumerate}
\item[1)] $1_{\inf_{k\geqslant n} A_k} = \inf \limits_{k\geqslant n} 1_{A_k}$.
\item[2)] $1_{\sup_{k\geqslant n} A_k} = \sup \limits_{k\geqslant n} 1_{A_k}$.
\item[3)] $1_{\cup_{n\geqslant 1} A_n} \leqslant \sum \limits_{n \geqslant 1} 1_{A_n}$.
\item[4)] $1_{\limsup A_n} = \limsup 1_{A_n}$.
\item[5)] $1_{\liminf A_n} = \liminf 1_{A_n}$.
\item[6)] $ 1_{A \triangle B} = 1_A + 1_B (\text{mod 2})$.
\end{enumerate}
