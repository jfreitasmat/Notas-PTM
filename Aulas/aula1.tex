\chapter[Aula 1]{Conjuntos e Limite de Sequências de Conjuntos}
\chaptermark{}

Nesta seção são introduzidos alguns fatos básicos da teoria
conjuntos. O ponto de vista adotado aqui é o da "teoria ingênua
de conjuntos" e não temos objetivos de discutí-la do ponto 
vista axiomático.

  
\section{Teoria Básica de Conjuntos}


Em geral, vamos trabalhar em um espaço que será denotado 
por $\Omega$. Assim, operações de conjuntos serão sempre 
consideradas com relação a este espaço. 
A coleção de todos os subconjuntos de $\Omega$ será 
denotada por $\mathcal{P}(\Omega)\equiv \{A: A\subset \Omega\}$ e 
será chamado de conjunto das partes de $\Omega$.
Em geral, usaremos as letras maiúsculas $A,B$ e etc. para denotar
um subconjunto arbitrário de $\Omega$. Quando quisermos nos referir
a uma coleção de subconjuntos de $\Omega$ usaremos letras maiúsculas 
caligráficas como $\mathcal{A}, \mathcal{B}$ e etc. Finalmente,
usaremos, na maioria das vezes, a notação $\omega\in\Omega$ 
para denotar um ponto do espaço $\Omega$.




\subsection{Operações de Conjuntos}

O complementar de um conjunto $A$ será 
denotado por $A^{c}= \{w \in \Omega; w \notin A\}$. 
Se $I$ é um conjunto de índices arbitrários e 
para cada $i\in I$ temos que $A_i\subset \Omega$ então 
a interseção e união dos conjuntos $A_i$'s sobre a coleção
$I$ são dados, respectivamente, por 
\begin{enumerate}
	\item[]  $\displaystyle\bigcap_{t \in T}{A_t}
							= 
						\{ w \in \Omega;\ w \in A_t,\ \forall t \in T\}$.
	\item[]  $\displaystyle\bigcup_{t \in T}{A_t}
							= 
						\{ w \in \Omega;\ w \in A_t,\ \text{para algum}\ t \in T \}$.
\end{enumerate}
%
%
%
Se $A \cap B = \emptyset$, dizemos que $A$ e $B$ são disjuntos.
Dizemos que uma sequência de conjuntos 
$A_1, A_2,\ldots$ é mutuamente disjunta (ou dois a dois disjunta) 
se $A_i \cap A_j = \emptyset$ sempre que $i \neq j$. 
Quando $A_1,A_2,\ldots $ for uma sequência mutuamente disjunta, 
vamos usar a notação abaixo para denotar a união dos conjuntos $A_n$'s
\[
	\displaystyle\bigcup_{n \geqslant 1}{A_n} = \sum_{n \geqslant 1} A_n.	
\]

\noindent
\textbf{Outras Notações:} Quando for conveniente 
usaremos as seguintes notações e convenções:
\begin{enumerate}
\item[$\blacklozenge$] 
$AB\equiv A \cap B$.

\item[$\blacklozenge$] 
$A \setminus B \equiv A \cap B^c$.

\item[$\blacklozenge$] 
Diferença Simétrica 
$A\triangle B = (A \setminus B) \cup (B\setminus A).$

\item[$\blacklozenge$] 
$\emptyset^c = \Omega$.

\item[$\blacklozenge$] 
$\Omega^c = \emptyset$.
\end{enumerate}




\begin{exercicio}[Associatividade] 
Mostre que para quaisquer subconjuntos $A,B$ e $C$ 
de $\Omega$ valem as seguintes igualdades:
\begin{enumerate}
	\item $(A\cup B)\cup C= A \cup ( B \cup C)$.
	\item $(A\cap B)\cap C= A \cap ( B \cap C)$.
\end{enumerate}
\end{exercicio}






\begin{exercicio}[Leis de de Morgan] 
Sejam $I$ um conjunto arbitrário de índices
e $A_i\subset \Omega$ para todo $i\in I$. Mostre que
as seguintes igualdades são válidas:
\begin{enumerate}
\item 
$
\left( \displaystyle\bigcup_{i \in I}{A_i} \right)^c 
= 
\displaystyle\bigcap_{i \in I}{{A_i}^c}
$.

\item
$
\left( \displaystyle\bigcap_{i \in I}{A_i} \right)^c 
= 
\displaystyle\bigcup_{i \in I}{A_i^c}
$.
%
\end{enumerate}
\end{exercicio}





\begin{exercicio}[Distributiva] 
Sejam $I$ um conjunto arbitrário de índices, 
$B\subset \Omega$ e $A_i\subset \Omega$ para todo $i\in I$. 
Mostre que as seguintes igualdades são válidas:
%
\begin{enumerate}
\item 
$
B \cap \left( \displaystyle\bigcup_{i \in I}{A_i} \right) 
= 
\displaystyle\bigcup_{i \in I}{(B\cap A_i)} 
$.
%
\item
$
B \cup \left( \displaystyle\bigcap_{i \in I}{A_i} \right) 
= 
\displaystyle\bigcap_{i \in I}{(B\cup A_i)} 
$.
\end{enumerate}
%
\end{exercicio}







\begin{definicao}[Função Indicadora]\label{def-funcao-indicadora}
	Seja $A \subseteq \Omega$. 
	A função indicadora \index{função!indicadora} 
	de $A$ é denotada por $1_A: \Omega \to \R$ e definida 
	como segue
	\[
		1_A(w) =
			\begin{cases}
				1, & \text{se}\ w \in A; \\
				0, & \text{caso contrário.}
			\end{cases}
	\]
\end{definicao}








\begin{observacao} 
	Da definição de função indicadora, podemos mostrar facilmente 
	as seguintes relações,
	as quais serão importantes ao longo do texto
	\begin{enumerate}
		\item 
		$1_A \leqslant 1_B \Leftrightarrow A \subseteq B$.

		\item
		$1_{A^c}= 1- 1_A$.
\end{enumerate}
\end{observacao}







\subsection{Limite de Conjuntos}

Seja $\{A_n\}$ uma sequência de subconjuntos de $\Omega$.
A partir desta sequência podemos definir outros 4 novos conjuntos
como segue:
\begin{enumerate}
\item[$\blacklozenge$] 
	$\inf \limits_{k\geqslant n} A_k 
	\equiv 
	\displaystyle\bigcap_{k=n}^{\infty}{A_k}$.

\item[$\blacklozenge$] 
	$\sup \limits_{k\geqslant n} A_k 
	\equiv 
	\displaystyle\bigcup_{k=n}^{\infty}{A_k}$.

\item[$\blacklozenge$] 
	$\liminf \limits_{n \to \infty} A_n 
	\equiv 
	\displaystyle\bigcup_ {n\geqslant 1} 
		\left(\displaystyle\bigcap_{k=n}^{\infty}{A_k} \right)
	$.

\item[$\blacklozenge$] 
	$\limsup \limits_{n \to \infty} A_n 
	\equiv 
	\displaystyle\bigcap_ {n\geqslant 1} 
		\left(\displaystyle\bigcup_{k=n}^{\infty}{A_k} \right)
	$.
\end{enumerate}






\begin{definicao}[Limite de uma Sequência de Conjuntos]
	Se uma sequência de conjuntos $B_n \subseteq \Omega$ é tal que 
	\[
		\liminf \limits_{n \to \infty} B_n 
		= 
		B
		=
		\limsup \limits_{n \to \infty} B_n, 
	\]	
	então dizemos que existe $\lim \limits_{n \to \infty} B_n = B$.
\end{definicao}




\begin{observacao}
Com um resultado sobre limite de sequências monótonas de conjuntos, 
mostraremos mais a frente que
		 \[
		 	\liminf \limits_{n\to \infty}A_n 
			=
			\lim \limits_{n\to \infty}\left(\inf \limits_{k\geqslant n}A_k \right).
		 \]
\end{observacao}








\begin{lema}
Seja $\{A_n\}$ uma sequência de subconjuntos de $\Omega$.
\begin{enumerate}
\item 
$\displaystyle\limsup_{n\to\infty} A_n 
= 
\{ w \in \Omega;\ \sum_{n \geqslant 1} 1_{A_n}(w)= \infty \}$.

\item  \ \\[-0.4cm] \hspace*{-0.35cm} % Arrumar esta gambiarra no futuro
$
\begin{array}{rl}
\displaystyle\liminf_{n\to\infty} A_n 
&= \{ w \in \Omega;\ w \in A_n \text{ para todo $n$ excepto uma quantidade finita} \} \\
&= \{w \in \Omega;\ \sum_{n\geqslant 1} 1_{A_n^c}(w) < \infty \} \\[0.2cm]
&= \{w \in \Omega;\ w \in A_n, \forall n \geqslant n_0(w) \}. 
\end{array} 
$

\end{enumerate}
\end{lema}

\begin{proof}
Prova do item $1$. Suponha que $w \in \limsup A_n$, então 
$w \in \cup_{k \geq n} A_k,
\forall n \in \N$, logo existe $k_n \geqslant n$ tal que $w \in A_{k_n}$. 
Assim
%
	\[
		\sum \limits_{n \geqslant 1} 1_{A_n}(w) 
		\geqslant 
		\sum \limits_{n\geqslant 1} 1_{A_{k_n}}(w) 
		= 
		\infty.
	\]
%
Reciprocamente, 
se 
$ w \in \{ w \in \Omega;\ \sum_{n \geqslant 1} 1_{A_n}(w)= \infty \}$, 
então para infinitos valores de $k$ temos que $w \in A_k$. 
Portanto $w \in \limsup A_n$.

A prova do item $2$ segue diretamente do item $1$. 
\end{proof}



\begin{exercicio} 
Seja $\{A_n\}$ uma sequência de subconjuntos de $\Omega$.
Mostre que as seguintes igualdades são verdadeiras:
%
\begin{enumerate}
\item $\liminf A_n \subseteq \limsup A_n$.
\item $\left( \liminf A_n \right)^c = \limsup A_n^c$. 
\end{enumerate}
\end{exercicio}







\begin{observacao} Este comentário pode ser omitido 
por leitores que nunca fizeram um curso introdutório 
de Probabilidade.
Seja $\{X_n,\ n\geqslant 0\}$ uma sequência de v.a.'s.
Uma das maneiras de mostrar que $X_n \to X$ q.c. 
é provar que   
\[
	\mathbb{P}( |X_n-X| > \epsilon \ \text{infinitas vezes} )=0,
\] 
em outras palavras, 
se denotamos por $A_n= \{ |X_n-X|> \epsilon\}$ 
então basta provar que $\mathbb{P} (\limsup A_n )=0$.
Voltaremos a este critério posteriormente e apresentaremos
sua prova no momento apropriado.
\end{observacao}







\begin{definicao}[Sequências Monótonas de Conjuntos]
Seja $\{A_n\}$ é uma sequência de conjuntos de $\Omega$. 
Dizemos que $\{A_n\}$ é monótona não-decrescente 
se $A_1 \subseteq A_2 \subseteq \ldots$.
Analogamente definimos sequência não-crescente.
Usaremos as notações $A_n \nearrow$ ou $A_n \uparrow$ 
(analogamente $A_n\searrow$ ou $A_n \downarrow$)
para indicar que $A_n$ é uma sequência 
não-decrescente (não-crescente).
\end{definicao}







\begin{proposicao}
 Suponha que $\{A_n\}$ é uma sequência monótona.
 \begin{enumerate}
 \item Se $A_n \nearrow$ então 
 	$\exists \lim \limits_{n \to \infty} A_n
 	= 
 	\displaystyle\bigcup_{n\geqslant 1} {A_n}$.
 	
 \item Se $A_n \searrow$ então 
 	$\exists \lim \limits_{n \to \infty} A_n
 	= 
 	\displaystyle\bigcap_{n\geqslant 1} {A_n}$.
 \end{enumerate}
\end{proposicao}


\begin{proof}
Vamos provar inicialmente o item 1. 
Neste caso queremos mostrar que 
%
\[
	\liminf_{n\to\infty} A_n
	= 
	\limsup_{n\to\infty} A_n
	=
	\displaystyle\bigcup_{n\geqslant 1} {A_n}.
\] 
%
Já que $A_j \subseteq A_{j+1}$
então $\cap_{k\geqslant n} {A_k}=A_n$.
Assim segue da definição que 
\[
\liminf_{n\to\infty} A_n 
= 
\bigcup_{n\geqslant 1} {A_n}.
\]
Usando a definição de $\limsup$ e que a interseção de 
uma sequência de conjuntos
está contida em qualquer elemento da sequência temos 
\[
	\limsup_{n\to\infty} A_n 
	=
	\displaystyle\bigcap_{n\geqslant 1} 
		\left(\displaystyle\bigcup_{k\geqslant n}{A_k} \right)
	\subseteq 
	\displaystyle\bigcup_{k\geqslant 1} {A_k} 
	=
	\liminf_{n\to\infty} A_n 
	\subseteq 
	\limsup_{n\to\infty} A_n.
\]
Para provar o item 2 basta proceder de maneira análoga feita acima 
e usar as Leis de De Morgan.
\end{proof}

\subsection{Relações de ``Dualidade''}
\begin{enumerate}
\item[1)] 
$1_{\inf_{k\geqslant n} A_k} = \inf \limits_{k\geqslant n} 1_{A_k}$.

\item[2)] 
$1_{\sup_{k\geqslant n} A_k} = \sup \limits_{k\geqslant n} 1_{A_k}$.

\item[3)] 
$1_{\cup_{n\geqslant 1} A_n} 
\leqslant 
\sum \limits_{n \geqslant 1} 1_{A_n}$.

\item[4)] 
$1_{\limsup A_n} = \limsup 1_{A_n}$.

\item[5)] 
$1_{\liminf A_n} = \liminf 1_{A_n}$.

\item[6)] 
$ 1_{A \triangle B} = 1_A + 1_B (\text{mod 2})$.
\end{enumerate}
