\chapter[Aula 4]{Espaços de Probabilidade}
\chaptermark{}





\section{Definições e Propriedades Básicas}

Um espaço de probabilidade é um tripla 
$(\Omega,\F,\P)$ onde 
\begin{itemize}
	\item 
	$\Omega$ é um conjunto chamado de espaço amostral.

	\item
	$\F$ uma $\sigma$-álgebra de subconjuntos de $\Omega$.
	Os elementos de $\F$ são chamados de eventos.
	
	\item $\P$ é uma medida de probabilidade, isto é, 
	uma função com domínio em $\F$ tomando valores em 
	$[0,1]$ satisfazendo 
		\begin{itemize}
			\item[i)] 
			$\P(E)\geq 0$ para todo $E\in F$.

			\item[ii)] 
			$P$ é $\sigma$-aditiva: Para qualquer
			sequência $\{E_n\}$ mutuamente disjunta 
			de eventos em $\F$ temos 
				\[
					\P\left( \bigcup_{n=1}^{\infty} E_n \right)
					=
					\sum_{n=1}^{\infty} \P(E_n).
				\]
			\item[iii)] 
			$\P(\Omega)=1$.
		\end{itemize}
\end{itemize}



Seguem das propriedade $\textrm{ii})$ e $\textrm{iii})$ acima que 
para qualquer evento $E\in\F$ vale a seguinte identidade
	\[
		P(E^c)=1-P(E).
	\]
De fato, 
	\[
		1=P(\Omega)= P(E\cup E^c)=P(E)+P(E^c).
	\]
Como toda medida de probabilidade é em particular uma medida então 
temos que 
	\[
		\P(\emptyset)=0.
	\] 
\begin{proposicao} Seja $(\Omega,\F,\P)$ um espaço de probabilidade.
Se $A,B$ são dois eventos arbitrários então 
	\[
		\P(A\cup B) = \P(A)+\P(B)-\P(A\cap B).
	\]
\end{proposicao}
\begin{proof}
Primeiro observamos que 
%	
	\begin{align*}
		\P(A) = \P(A\cap B^c)+\P(A\cap B) 
		\\
		\P(B) = \P(B\cap A^c)+\P(B\cap A) 
	\end{align*}
Note que podemos escrever $A\cup B = (A\cap B^c) \cup (B\cap A^c) \cup (A\cap B)$.
Tomando probabilidade em ambos lados desta igualdade 
e usando as identidades acima, temos
	\begin{align*}
	\P(A\cup B) 
	&=
	\P(A\cap B^c) + \P(B\cap A^c) + \P(A\cap B)
	\\
	&=
	[\P(A)-\P(A\cap B)] + [\P(B)-\P(A\cap B)] + P(A\cap B)
	\\
	&=
	\P(A)+\P(B)-\P(A\cap B).
	\end{align*}


\end{proof}

\begin{teorema}[Fórmula de Inclusão-Exclusão]
Para quaisquer eventos $E_1,\ldots,E_n$ temos 
	\begin{align*}
		\P\left( \bigcup_{j=1}^{n} E_j \right)
		&=
		\sum_{j=1}^n \P(E_j)
		-
		\sum_{1\leq i<j\leq n} \P(E_i\cap E_j) 
		+
		\sum_{1\leq i<j<k\leq n} \P(E_i\cap E_j\cap E_k)-\ldots
		\\[0.3cm]
		&\hspace*{0.5cm} \ldots(-1)^n \P(E_1\cap\ldots\cap E_n).
	\end{align*}
\end{teorema}

\begin{proof}
 Podemos provar o teorema por indução. Para $n=2$ o teorema
 segue diretamente da proposição acima. Para dar o passo de 
 indução basta escrever $\cup_{j=1}^{k+1} E_j$ 
 como $\cup_{j=1}^{k}E_j\cup E_{k+1}$, usar o caso $n=2$ 
 para calcular a probabilidade desta decomposição, em seguida usar 
 a hipótese de indução para 
 \[	
 \P \left( \bigcup_{j=1}^{k} E_j \right)
 \qquad
 \text{e}
 \qquad
 \P \left( \bigcup_{j=1}^{k} (E_j\cap E_{k+1}) \right)
 \]
 e reorganizar os termos de ambas as somas. Observe 
 por exemplo, que as probabilidades de interseções de pares 
 com $E_j$'s com índice menores ou iguais a $k$ virão 
 da primeira probabilidade e interseção de pares de
 eventos $E_j$'s onde um dos eventos é $E_{k+1}$ virá 
 da segunda soma e assim por diante.
\end{proof}

Uma aplicação deste Teorema é a obtenção das
chamadas {\bf Desigualdades de Bonferroni} \index{Desigualdade!Bonferroni}.
A prova destas desigualdades segue simplesmente de negligenciarmos 
os resto. Dois exemplos de tais desigualdades são mostrados
abaixo:
\[ 
\begin{array}{c}
	\displaystyle
	\P \left( \bigcup_{j=1}^{n} E_j \right) 
	\leq 
	\sum_{j=1}^n P(E_j)
	%
	%
	\\[0.5cm]
	%
	%
	\displaystyle
	\P \left( \bigcup_{j=1}^{n} E_j \right) 
	\geq
	\sum_{j=1}^n \P(E_j) - \sum_{1\leq i<j\leq n} \P(E_i\cap E_j)	
\end{array}
\]

Outra propriedade importante é a {\bf Monotonicidade} de $\P$, isto é, 
para quaisquer eventos $A,B$ tais que $A\subset B$ temos 
$\P(A)\leq \P(B)$. De fato, 
$\P(B)=\P(B\cap A)+\P(B\cap A^c)= \P(B)+\P(B\setminus A)\geq \P(A)$.

Toda medida de probabilidade é $\sigma$-subaditiva, ou seja, 
para quaisquer eventos $E_1,E_2,\ldots$ temos 
	\[
		\P\left( \bigcup_{n=1}^{\infty} E_n\right)
		\leq 
		\sum_{n=1}^{\infty} \P(E_n).
	\]
Para verificar que esta propriedade é verdadeira basta observar que 
\[
	\bigcup_{n=1}^{\infty} E_n = E_1+E_1^c\cap E_2 + E_1^c\cap E_2^c\cap E_3 +\ldots
\]
aplicar a $\sigma$-aditividade de $\P$ e em seguida usar a monotonicidade 
obtendo
	\begin{align*}
	\P\left( \bigcup_{n=1}^{\infty} E_n\right)
	&=
	\P(E_1)+\P(E_1^c\cap E_2) + \P(E_1^c\cap E_2^c\cap E_3) +\ldots
	\\
	&\leq
	\P(E_1) + \P(E_2) + \P(E_3) +\ldots
	\end{align*}
	
\begin{teorema}[Continuidade da Probabilidade]
	Sejam $(\Omega,\F,\P)$ um espaço de Probabilidade e
	$\{E_n\}$ uma sequência de eventos neste espaço.
	\begin{enumerate}
		\item
		Se $E_n\nearrow E$ então 
		$\displaystyle\lim_{n\to\infty}\P(E_n)=\P(E)$.
		\item 
		Se $E_n\searrow E$ então 
		$\displaystyle\lim_{n\to\infty}\P(E_n)=\P(E)$.		
	\end{enumerate}
\end{teorema}

\begin{proof}
Vamos provar o primeiro item do Teorema. 
Suponha que $E_1\subset E_2\subset \ldots\subset E_n\subset \ldots$
e defina a seguinte sequência de eventos 
\[
B_1= E_1,\ B_2= E_2\setminus E_2,\ \ldots,\  B_n=E_n\setminus E_{n-1}.
\]
Por construção a sequência $\{B_n\}$ é uma sequência de eventos
mutuamente disjunta e além do mais 
\[
\bigcup_{j=1}^n B_j = E_n, \quad 
\bigcup_{j=1}^{\infty} B_j = \bigcup_{j=1}^{\infty} E_j = E.
\]
Da $\sigma$-aditividade segue que 
\begin{align*}
	\P(E)
	&=
		\P\left( \bigcup_{n=1}^{\infty} E_n\right)
		=
		\sum_{n=1}^{\infty}\P(B_n)
		=
		\lim_{n\to\infty} \sum_{j=1}^n \P(B_j)
	\\[0.3cm]
	&=
		\lim_{n\to\infty}\P\left( \bigcup_{j=1}^{n} B_j\right)
		=
		\lim_{n\to\infty}\P\left( E_n\right).
\end{align*}

Para provar o segundo item note que se $E_n\searrow E$ 
então $E_n^c \nearrow E^c$. Aplicando o resultado que 
acabamos de provar temos que 
	\[
		1-\P(E)
		=
		\P(E^c)		
		=
		\lim_{n\to\infty} \P(E_n^c)
		=
		\lim_{n\to\infty} [1-\P(E_n)]		
	\]
de onde segue o resultado.
\end{proof}

Vamos provar agora um lema de continuidade para sequências 
que não são necessariamente monótonas. 
Este lema será chamado de lema de Fatou para funções indicadoras
e a razão para isto ficará clara futuramente.

\begin{lema}[Lema de Fatou para Funções Indicadoras]
Seja $(\Omega,\F,\P)$ um espaço de probabilidade e 
$\{E_n\}$ uma sequência de eventos.
\begin{enumerate}
	\item
	$\displaystyle
	\P(\liminf_{n\to\infty} E_n)
	\leq 
	\liminf_{n\to\infty} \P(E_n)
	\leq
	\limsup_{n\to\infty} \P(E_n)
	\leq
	\P(\limsup_{n\to\infty} E_n)$.
	
	\item Se $E_n\to E$ então $\P(E_n)\to \P(E)$.
\end{enumerate}
\end{lema}


\begin{proof}
Primeiro vamos assumir que 1 é válido e vamos mostrar 2.
Suponha que $E_n\to E$. Então 
\[
 \liminf_{n\to\infty} E_n = \limsup_{n\to\infty} E_n = E.
\]
Aplicando 1 temos que 
	\begin{align*}
	\P(E)
	&=\displaystyle
		\P(\liminf_{n\to\infty} E_n)
		\leq 
		\liminf_{n\to\infty} \P( E_n)
		\leq
		\limsup_{n\to\infty}\P(E_n)
		\leq
		\P(\limsup_{n\to\infty} E_n)
		=
		P(E).
	\end{align*}
O que prova que $\P(E_n)\to P(E)$.

Passamos agora a prova do item 1. Por definição 
do $\liminf$, continuidade da probabilidade
temos 
\begin{align*}
	\P(\liminf_{n\to\infty} E_n)
	&=
	\P\left( \lim_{n\to\infty} \bigcap_{k\geq n}E_k \right)
	\\
	&=
	\lim_{n\to\infty} \P\left( \bigcap_{k\geq n}E_k \right)
\end{align*}
Por monotonicidade temos que 
	\[
		\P\left( \bigcap_{k\geq n}E_k \right)
		\leq 
		\P(E_n).
	\] 
Assim se tomamos $\liminf$ em ambos os lados e usamos as igualdades acima,
obtemos a seguinte desigualdade
	\begin{align*}
	\P(\liminf_{n\to\infty} E_n)
	\leq 
	\liminf_{n\to\infty}\P(E_n).
\end{align*} 

De maneira análoga, temos 
\begin{align*}
	\P(\limsup_{n\to\infty} E_n)
	&=
	\P\left( \lim_{n\to\infty} \bigcup_{k\geq n}E_k \right)
	\\[0.2cm]
	&=
	\lim_{n\to\infty} \P\left( \bigcup_{k\geq n}E_k \right)
	\\[0.3cm]
	&\geq 
	\limsup_{n\to\infty} \P(E_n).
\end{align*}
\end{proof}



\section{Função Distribuição}

\begin{exemplo}
Seja $\Omega=\R$ (conjunto dos números reais) e $\P$ uma medida de probabilidade em $\R$.
Defina $F:\mathbb{R}\to [0,1]$ por 
	\[
		F(x) =\P((-\infty,x]).
	\]
Então 
\begin{enumerate}
	\item	
	$F$ é contínua a direita.
	
	\item
	$F$ é monótona não-decrescente	
		
	\item 
	Existem os seguintes limites 
		\begin{itemize}
			\item 
 			$\displaystyle\lim_{x\to+\infty} F(x)=1$		
			
			\item
			$\displaystyle\lim_{x\to -\infty} F(x)=0$.
		\end{itemize}
\end{enumerate}
\end{exemplo}
\begin{proof}
	Vamos mostrar primeiro o item 2. Se $x<y$ então temos que 
	$(-\infty,x]\subset (-\infty,y]$. O resultado segue diretamente 
	da monotonicidade de $\P$ e da definição de $F$.
	
	Passamos agora para a prova do item 3. 
	Se $x_n\to\infty$ quando $n\to\infty$, 
	segue da monotonicidade de $F$ que 
	existe $\lim_{n\to\infty} F(x_n)$. 
	Usando a definição de $F$ e as propriedades de $\P$ temos 
	as seguintes igualdades:
	\begin{align*}
	\lim_{x_n\to\infty} F(x_n)
	&=
	\lim_{n\to\infty} \P((-\infty, x_n])
	\\
	&=
	\P \left( \lim_{n\to\infty} (-\infty,x_n] \right)
	\\
	&=
	\P \left( \bigcup_{n=1}^{\infty} (-\infty,x_n] \right)
	\\
	&=
	\P((-\infty,+\infty)=\P(\mathbb{R})=\P(\Omega)=1.
	\end{align*}
Já que $x_n$ é uma sequência arbitrária tendendo 
a $+\infty$ podemos concluir que existe 
$\lim_{x\to+\infty} F(x)=1$.  O cálculo do outro limite 
do item 3 é análoga, bastando agora usar a continuidade no 
vazio. 

Para provar o item 1 temos que mostrar que sempre que 
$x_n \downarrow x$ temos que $F(x_n) \downarrow F(x)$.
Mas isto é consequência imediata da propriedade de continuidade
de $\P$ e de que 
	\[
		(-\infty,x_n] \searrow (-\infty,x].
	\]
\end{proof}







\begin{definicao}[Função Distribuição]
Uma função $F:\mathbb{R}\to [0,1]$ satisfazendo 
as seguintes propriedades:
\begin{enumerate}
	\item	
	$F$ é contínua a direita.
	
	\item
	$F$ é monótona não-decrescente	
		
	\item 
	Existem os seguintes limites 
		\begin{itemize}
			\item 
 			$\displaystyle\lim_{x\to+\infty} F(x)=1$		
			
			\item
			$\displaystyle\lim_{x\to -\infty} F(x)=0$.
		\end{itemize}
\end{enumerate}
é chamada de uma função distribuição.
\end{definicao}






\section{Construção de $\P_{F}$}

O objetivo nesta seção é construir uma 
medida de probabilidade em $\R$ com uma 
dada função distribuição $F$. 
Para isto vamos usar a medida de Lebesgue que 
nesta seção será denotada por $\lambda$.


