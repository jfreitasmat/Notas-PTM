\chapter*{Lista 1}
\addcontentsline{toc}{chapter}{Lista 1}
\chaptermark{}
%%%%%%%%%%%%%%%%%%%%%%%%%%%%%%%%%%%%%%%%%%%%%%%%

% Inicio da Lista de Exercícios 
\begin{enumerate}[leftmargin=*]


\item 
Seja $\Omega$ um espaço não vazio. 
Seja $\F_0$ a coleção de todos os 
subconjuntos $E\subset\Omega$ 
tais que $E$ ou $E^c$ é vazio.
Considere a função de conjuntos $P:\F_0\to [0,1]$
definida por 
	\[
		P(E) =
		\begin{cases}
		0,&\text{se}\ E \ \text{é finito};
		\\
		1,&\text{se}\ E^c \ \text{é finito}.
		\end{cases}
	\]
	\begin{itemize}
		\item[a)]
		Mostre que $\F_0$ é uma álgebra de conjuntos.
		
		\item[b)] 
		Assumindo que $\Omega$ é um conjunto infinito enumerável,
		mostre que $P$ é finitamente aditiva, mas não é $\sigma$-aditiva.
		
		\item[c)]
		Assumindo que $\Omega$ é não-enumerável, mostre que $P$ 
		é $\sigma$-aditiva em $\F_0$.
		
	\end{itemize}





\item 
Seja $(\Omega,\F,\P)$ um espaço de probabilidade.
Se para todo $n\in\N$ o evento 
$B_n\subset A_n$ então mostre que 
	\[
		\P\left( \bigcup_{n=1}^{\infty} A_n \right)
		-\P\left( \bigcup_{n=1}^{\infty} B_n  \right)	
		\leq 
		\sum_{n=1}^{\infty} \Big( \P(A_n)-\P(B_n) \Big)	
	\]





\item 
Seja $(\Omega,\F)$ um espaço de mensurável. 
Suponha que $\P$ é uma medida de probabilidade
definida em $\F$ e que existe um conjunto $A\subset\Omega$
tal que $A\notin\F$. Considere a $\sigma$-álgebra 
$\F_1 = \sigma(\F,A)$. Mostre que $\P$ admite uma extensão 
a uma medida de probabilidade $\P_1$ definida em $\F_1$. 





\item 
Seja $\P$ uma medida de probabilidade em $\mathscr{B}(\R)$.
Mostre que para qualquer $B\in \mathscr{B}(\R)$ e $\varepsilon>0$ 
dado, existe uma união finita de intervalos $A$ tal que 
	\[
		\P(A\bigtriangleup B) < \varepsilon.
	\]
\noindent{ Dica.} Defina a seguinte coleção:
\[ 
	\mathscr{C}
	=
	\{
		B\in\mathscr{B}(\R): \forall \varepsilon>0,
		\exists A_{\varepsilon} (\text{união finita de interv.})
		\ \text{tal que} \P(A_{\varepsilon} \bigtriangleup B)<\varepsilon
	\}
\]



\item 
Dizemos uma sequência de eventos $\{A_n\}$ 
em um espaço de probabilidade
$(\Omega,\F,\P)$ é quase disjunta se 
$\P(A_i\cap A_j)=0$ se $i\neq j$. 
Para tais eventos mostre que 
\[
	\P\left( \bigcup_{n=1}^{\infty} A_n \right)
	=
	\sum_{n=1}^{\infty} \P(A_n).
\]







\item Sabemos que $\P_1=\P_2$ em $\F$ dado que $\P_1=\P_2$
em $\mathcal{C}$ e $\mathcal{C}$ é $\pi$-sistema 
que gera $\F$. 
Mostre que a hipótese de $\mathcal{C}$ ser um $\pi$-sistema
não pode ser removida, através do seguinte exemplo:
considere $\Omega=\{a,b,c,d\}$ e que 
\[
	\P_1(\{a\}) = \P_1(\{d\}) = \P_2(\{b\}) = \P_2(\{c\}) = \frac{1}{6}
\]
e
\[
	\P_1(\{b\}) = \P_1(\{c\}) = \P_2(\{a\}) = \P_2(\{d\}) = \frac{1}{3}.
\]
Tome $\mathcal{C}=\{ \{a,b\}, \{d,c\}, \{a,c\}, \{b,d\} \}$.












\item 
{\bf Definição.} \index{Conjuntos!\P equivalentes}
Seja $(\Omega,\F,\P)$ um espaço de probabilidade. 
Dizemos que dois conjuntos $A,B\in \F$ são equivalentes
se $\P(A\bigtriangleup B)=0$. A classe de equivalência
do envento $A$ é dada por 
\[ 
	[A] = \{B\in \F: \P(A \bigtriangleup B)=0 \}.
\]
Desta forma podemos decompor $\F$ em classes de equivalência.
\\
{\bf Definição.} Um átomo \index{Átomo} em um espaço 
de probabilidade $(\Omega,\F,\P)$ é definido como sendo
um conjunto $A\in\F$ tal que $P(A)>0$ e se 
$B\subset A$ e $B\in \F$, então $\P(B)=0$ ou 
$P(A\setminus B)=0$. 
Além do mais o espaço de probabilidade é dito não-atômico
se ele não possui átomos, isto é, se $A\in F$ e 
$P(A)>0$ então existe pelo menos um evento $B\in \F$ tal que
$B\subset A$ e $0<P(B)<P(A)$.

	\begin{itemize}
		\item[a)]
		Se $\Omega=\R$ e $\P$ é determinada por uma 
		função distribuição $F$, mostre que os átomos
		são $\{x\in\R: F(x)-F(x-)>0\}$.
				
		\item[b)]
		Mostre que o espaço de probabilidade  
		$(\Omega,\F,\P)=((0,1],\mathscr{B}((0,1]),\lambda)$,
		onde $\lambda$ é a medida de Lebesgue, é não-atômico. 

		\item[c)]
		Mostre que se $A$ e $B$ são átomos tais que
		$\P(A\bigtriangleup B)>0$ então temos que
		$\P((A\cap B) \bigtriangleup \emptyset)=0$.


		\item[d)] 
		Um espaço de probabilidade contém no máximo 
		uma quantidade enumerável de átomos. 
		\\
		(Dica. Qual é o número máximo de átomos que o
		espaço pode conter tendo probabilidade pelo menos $1/n$\ ?)
		
		
		\item[e)]
		Se um espaço de probabilidade $(\Omega,\F,\P)$
		não contém átomos, então para todo $a\in (0,1]$
		existe pelo menos um conjunto $A\in\F$ tal que 
		$\P(A)=a$. 
		\\
		(Uma maneira de provar este fato é usando o Lema de Zorn.)
	\end{itemize}






\item
Suponha que $\{E_n\}$ é uma sequência de eventos tal que
para todo $n\in\mathbb{N}$ temos $\P(E_n)=1$. Mostre que 
\[	
	\P\left(  \bigcap_{n=1}^{\infty} E_n\right) =1 
\]






\item
Suponha que $\mathcal{C}$ seja uma coleção de subconjuntos de 
$\Omega$ e que $B\subset \Omega$ é um evento satisfazendo
$B\in\sigma(\mathcal{C})$.
Mostre que existe uma coleção enumerável $\mathcal{C}_{B}$
tal que $B\in \sigma(\mathcal{C}_{B})$.
\\
(Dica. Considere o conjunto 
$\mathcal{G}=\{ B\subset\Omega: \exists\ \text{uma coleção enumerável}\ 
\mathcal{C}_B\subset \mathcal{C}\ 
\text{tal que}\ B\in\sigma(\mathcal{C}_{B}) \}
$
e mostre que $\mathcal{G}$ é uma $\sigma$-álgebra contendo
$\mathcal{C}$.)







\item 
Se $\{E_k\}$ é uma coleção de eventos tal que 
	\[
		\sum_{k=1}^{n} \P(E_k) > n-1
	\]
então 
\[	
	\P\left(  \bigcap_{n=1}^{n} E_n\right) >0. 
\]	






\item 
Mostre que se $F$ é uma função distribuição, 
então $F$ tem no máximo uma quantidade enumerável 
de descontinuidades.






\item 
Seja $F$ uma função distribuição e considere as seguintes
funções $G,H:(0,1)\to \mathbb{R}$ dadas por 
	\[
		G(y)= \inf\{t\in\mathbb{R}: y\leq F(t) \}
		\quad
		\text{e}
		\quad
		H(y)= \inf\{t\in\mathbb{R}: y < F(t) \}
	\]
	\begin{itemize}
		\item[a)	]
		Mostre que $G$ é contínua a esquerda e que $H$
		é contínua a direita.  

		\item[b)] 
		$\lambda \{t\in (0,1]: G(t)\neq H(t) \} =0$, 
		onde $\lambda$ é a medida de Lebesgue. 
	\end{itemize}







\item 
Suponha que $F$ é uma função distribuição contínua 
em $R$. Mostre que $F$ é uniformemente contínua.




\item 
Seja $\{E_n\}$ uma sequência de eventos e defina
\[
\begin{array}{c}
	\displaystyle \mathcal{S}_{1} 
	= \sum_{i=1}^{n} \P(E_i) 
	\\[0.5cm]
	%
	\displaystyle \mathcal{S}_{2} 
	= \sum_{1\leq i<j\leq n} \P(E_i\cap E_j) 
	\\[0.5cm]
	%
	\displaystyle\mathcal{S}_{3} 
	= \sum_{1\leq i<j<k\leq n} \P(E_i\cap E_j\cap E_k ) 
	\\[0.3cm]
	\vdots
\end{array}
\]
	Mostre que a probabilidade $(1\leq m\leq n)$
		\[
			p(m) = \P\left( \sum_{i=1}^n 1_{A_i} = m \right)
		\]
	de $m$ eventos ocorrerem é dada por 
		\[
			\begin{array}{rcl}
			p(m) 
			&=&\displaystyle
			\mathcal{S}_{m+1} 
			- 
			\binom{m+1}{m}\mathcal{S}_{m+1}
			+
			\binom{m+2}{m}\mathcal{S}_{m+2}
			-
			\ldots
			\\[0.7cm]
			&&\displaystyle
			+ (-1)^{n-m} \binom{n}{m}\mathcal{S}_{m+2}			
			\end{array}
		\]



































\end{enumerate}
%  Fim da Lista de Exercícios



