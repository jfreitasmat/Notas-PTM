\chapter*{Lista 1}
\addcontentsline{toc}{chapter}{Lista 1}
\chaptermark{}
%%%%%%%%%%%%%%%%%%%%%%%%%%%%%%%%%%%%%%%%%%%%%%%%

% Inicio da Lista de Exercícios 
\begin{enumerate}[leftmargin=*]


\item 
Seja $\Omega$ um espaço não vazio. 
Seja $\F_0$ a coleção de todos os 
subconjuntos $E\subset\Omega$ 
tais que $E$ ou $E^c$ é vazio.
Considere a função de conjuntos $P:\F_0\to [0,1]$
definida por 
	\[
		P(E) =
		\begin{cases}
		0,&\text{se}\ E \ \text{é finito};
		\\
		1,&\text{se}\ E^c \ \text{é finito}.
		\end{cases}
	\]
	\begin{itemize}
		\item[a)]
		Mostre que $\F_0$ é uma álgebra de conjuntos.
		
		\item[b)] 
		Assumindo que $\Omega$ é um conjunto infinito enumerável,
		mostre que $P$ é finitamente aditiva, mas não é $\sigma$-aditiva.
		
		\item[c)]
		Assumindo que $\Omega$ é não-enumerável, mostre que $P$ 
		é $\sigma$-aditiva em $\F_0$.
		
	\end{itemize}





\item 
Seja $(\Omega,\F,\P)$ um espaço de probabilidade.
Se para todo $n\in\N$ o evento 
$B_n\subset A_n$ então mostre que 
	\[
		\P\left( \bigcup_{n=1}^{\infty} A_n \right)
		-\P\left( \bigcup_{n=1}^{\infty} B_n  \right)	
		\leq 
		\sum_{n=1}^{\infty} \Big( \P(A_n)-\P(B_n) \Big)	
	\]





\item 
Seja $(\Omega,\F)$ um espaço de mensurável. 
Suponha que $\P$ é uma medida de probabilidade
definida em $\F$ e que existe um conjunto $A\subset\Omega$
tal que $A\notin\F$. Considere a $\sigma$-álgebra 
$\F_1 = \sigma(\F,A)$. Mostre que $\P$ admite uma extensão 
a uma medida de probabilidade $\P_1$ definida em $\F_1$. 





\item 
Seja $\P$ uma medida de probabilidade em $\mathscr{B}(\R)$.
Mostre que para qualquer $B\in \mathscr{B}(\R)$ e $\varepsilon>0$ 
dado, existe uma união finita de intervalos $A$ tal que 
	\[
		\P(A\bigtriangleup B) < \varepsilon.
	\]
\noindent{ Dica.} Defina a seguinte coleção:
\[ 
	\mathscr{C}
	=
	\{
		B\in\mathscr{B}(\R): \forall \varepsilon>0,
		\exists A_{\varepsilon} (\text{união finita de interv.})
		\ \text{tal que} \P(A_{\varepsilon} \bigtriangleup B)<\varepsilon
	\}
\]



\item 
Dizemos uma sequência de eventos $\{A_n\}$ 
em um espaço de probabilidade
$(\Omega,\F,\P)$ é quase disjunta se 
$\P(A_i\cap A_j)=0$ se $i\neq j$. 
Para tais eventos mostre que 
\[
	\P\left( \bigcup_{n=1}^{\infty} A_n \right)
	=
	\sum_{n=1}^{\infty} \P(A_n).
\]







\item Sabemos que $\P_1=\P_2$ em $\F$ dado que $\P_1=\P_2$
em $\mathcal{C}$ e $\mathcal{C}$ é $\pi$-sistema 
que gera $\F$. 
Mostre que a hipótese de $\mathcal{C}$ ser um $\pi$-sistema
não pode ser removida, através do seguinte exemplo:
considere $\Omega=\{a,b,c,d\}$ e que 
\[
	\P_1(\{a\}) = \P_1(\{d\}) = \P_2(\{b\}) = \P_2(\{c\}) = \frac{1}{6}
\]
e
\[
	\P_1(\{b\}) = \P_1(\{c\}) = \P_2(\{a\}) = \P_2(\{d\}) = \frac{1}{3}.
\]
Tome $\mathcal{C}=\{ \{a,b\}, \{d,c\}, \{a,c\}, \{b,d\} \}$.

\end{enumerate}
%  Fim da Lista de Exercícios



